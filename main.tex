\documentclass{article}
\usepackage[utf8]{inputenc}

%\title{FSG Driverless Rules}
%\author{Javi }
%\date{August 2020}
%
\begin{document}
%
%\maketitle
%
%\section{Introduction}

A Administrative regulations\\

A 2 Vehicle elegibility\\

A 2.3 Base Vehicles\\

A 2.3.1 In derogation from A2.2, reconfigured vehicles from any year may participate in the DV class. This includos vehicles that have been participating in the DV class in the previous year if there are significant changes in the autonomous system.\\

A 2.3.2 All Driverless Vehicles (DVs) must be fully compliant with the current version of these rules. Exceptions are granted for rules T 3.4.3, T 3.7.4 last two sentences, T 3.9.5 second and lasr sentence, and T 4.5.1. T 4.5.3 applies only if lap belts and anti-submarine belts are attached to the same attachment point.\\

A 4 General Requirements for Teams & Participants\\

A 4.9 Autonomous System Responsible (ASR)\\

A 4.9.1 Every participating team must appoint at least one ASR for the competition. This person is responsible for all autonomous operations of the vehicle during the competition which includes any work on the autonomous system as well as racing and testing.\\

A 4.9.2 For driverless vehicles with an electric drivetrain the ASR must fulfill A 4.8 and therefore replaces the ESO. The team may not register additional ESOs.\\

A 4.9.3 The ASR is the only person in the team who is allowed to declare the autonomous system safe, so that work on any system of the vehicle may be performed by the team, or the vehicle may be operated in manual or autonomous mode.\\

A 4.9.4 The ASR must be a valid team member, wich meand that he/she must have student status, see A 4.3.\\

A 4.9.5 The ASR must accompany the vehicle whenever it is operated or moved around at the competition site.\\

A 4.9.6 If only one ASR is named by the team, this ASR may not be a driver.\\

A 4.9.7 The ASR must be properly qualified to handle the autonomous syystem and to understand and deal with problems and failures. A bachelor degree in computer science, electrical engineering, mechatronics, automation engineering, robotics or similar is a sufficient qualification.\\

A 5 Documentation & Deadlines\\

A 5.1 Required Documentation and Forms\\

A 5.1.1 The following documents and forms must be submitted bi the action deadlines ddefined in the competition handbook:\\
	- Group A:\\
		IAD, SE3D, SES, SESA, AAIR & ASF\\
	- Group B:\\
		BPPV, CRD, DSS, EDR, ADR\\
	- Group C:\\
		TMD, MU, FTO, ASRQ\\
	- Grupo D:\\
		VSV\\

A 5.6 Vehicle Status Video (VSV)\\

A 5.6.3 The video must fulfill the following criteria:\\
	- Continuous video from a third person view - no assembled sequences.\\
	- Vehicle must be clearly visible (light, video resolution, frames and frequency).\\
	- Vehicle must run under its own power.\\
	- Driving in a clearly separated and/or protected area (A 6.4 applies).\\
	- Vehicle must be presented in ready-to-race conditions incl. body work.\\
	- Driver must wear all required equipment as specified in T 13.3.\\
	- Vehicle must drive without a driver.\\
	- Autonomous System Status Indicator (ASSI) must be clearly vissible in the video.\\
	- In addition to the third person view, an onboard view and a visualization of the vehicle's environment perception and path planning must be shown in split screen. All parts must be time syncronized.\\
	- At the end of the VSV, the vehicle must be stopper by an emergency brake maneuver (see DV 3).\\
	- Must not exceed a length of 45 seconds. File size may be limited, see the competition handbook.\\
	- File format must be common e.g. avi, mpg, mp4, wmv.\\

A 6 General Rules\\
	
A 6.3 Team Briefing\\

A 6.3.2 Drivers that want to operate a Driverless Vihicle (DV) in manual mode on the test track are required to attend the team briefing as well.\\

A 6.3.3 All ASR are required to attend the team briefing.\\

A 6.7 Vehicle Movement\\

A 6.7.4 Driverless vehicles must also have their autonomous system (see deficition on section DV 2.2) deactivated when being moved around the paddok. The detachable handle or key of the Autonomous System Master Switch (ASMS) must be completely removed and kept by an ASR. The lockout/tagout function of the ASMS, see DV 2.2.8, must be used.\\

T General Technical Requirements\\

T 4 Cockpit\\

T 4.2 Cockpit Internal Cross Section\\

T 4.2.4 To allow for the steering actuator a reduced-height template (reduced by 50 mm, shown in figure 11) may be used for a section measuring 200 mm horizontally along the template's path (compare T4.2.1).\\

T 4.2.5 The additional space allowed by T 4.3.4 and T 4.2.4 may only be used for steering, braking and clutch actuators. When the actuators are removed, the standard templates must fit into the cockpit.\\

T 4.3 Percy (95th male)\\

T 4.3.4 The figure has to be positioned in the vehicle as follows:\\
	- The seat adjusted to the frontmost position\\
	- The pedals adjusted to the frontmost position\\
	- The bottom 200 mm circle placed on the seat bottom. The distance between the center of the circle and the rearmost actuation face of pedals must be minimum 915 mm.\\
	- The distance from center of circle and pedals , as mentioned above, may be reduced to 865 mm but only for placement of automated brake, steering or clutch actuators in front of the pedals.\\
	- The middle circle positioned on the seat back\\
	- The upper 300 mm circle positioned 25 mm away from the head restraint.\\
	
T 6 Brake System\\

T 6.1 Brake System - General\\

T 6.1.4 In autonomous mode, it is allowed to use "brake-by-wire". In manual mode, T 6.1.1 applies.\\

T 11 Electrical Components\\

T 11.2 Master Switches\\

T 11.2.5 Master switches must be mounted next to each other.\\

T 11.9 System Critical Signals (SCSs)\\

T 11.9.1 SCS are defined as all electric signals which\\
	- Influence actions on the shutdown circuit, see CV 4.1 and EV 6.1.\\
	- Influence the wheel torque.\\
	- Influence indicator according to DV 3.2.7.\\

T 13 Vehicle and Driver Equipment\\

T 13.5 Camera Mounts\\

T 13.5.1 The mounts for video/photographic cameras must be of a safe and secure design:\\
	- All camera installations must be approved at technical inspection.\\
	- Helmet mounted cameras hare prohibited.\\
	- The body of any camera or recording unit must be secured at a minimum of two points on different sides of the camera body. If a tether is used to restrain the camera, the tether length must be limited so that the camera cannot contact the driver.\\
	Cameras used as input sensors for driverless vehicles are exempted and have to follow DV 4 instead.\\

CV Internal Combustion Engine Vehicles\\

CV 1 Internal Combustion Engine Powertrains\\

CV 1.2 Starter\\

CV 1.2.2 The vehicle must be equipped with an additional engine start button next to the LVMS, see T 11.3, that can be easily actuated from outside the vehicle. Using the external engine start button, the engine may only start if\\
	- the ASMS (see DV 2.2) is switched on and\\
	- the gearbox is in neutral.\\
	
CV 1.2.3 There must be a green ligth next to the engine start button, that indicates that the gearbox is in neutral. It must be marked with a letter "N". This letter must have a minimum height of 25 mm.\\

CV 1.2.4 The autonomous system must not be able to (re-)start the engine.\\

CV 1.6 Electronic Throttle Control (ETC)\\

CV 1.6.2 Any DV with internal combustion engine is assumed to have an ETC.\\

DV Driverless Vehicles\\

DV 1 Vehicle Requirements and Restrictions\\

DV 1.1 Base Vehicle\\
	Additions to the vehicle's general requirements and restrictions are marked and given in T, and CV or EV depending on the implemented drivetrain. Please also refer to rule A 2.3.\\
	The following definitions apply to Internal Combustion Engines to maintain the same wording as for Electric Vehicles.\\
	
DV 1.1.1 Ready-to-Drive (R2D) - Engine is running and a gear is engaged.\\

DV 1.1.2 TS active - Engine is running but gearbox is in neutral (also assumed for TS not active).\\

DV 1.1.3 TS activation button - The engine start button is the equivalent.\\

DV 1.2 Wireless communication\\

DV 1.2.1 It is prohibited to change parameters, send commands or make any software changes by wireless communication. Receiving information from the vehicle via one-way-telemetry is allowed. During dynamic events, wireless communication may be limmited and an uninterfered and reliable wireless connection is not guarranteed by the officials.\\

DV 1.2.2 The only device that is allowed to send commands by wireless communication is the Remote Emergency System (RES) described in DV 1.4.\\

DV 1.2.3 (D)GPS may be used, but there will be no space to securely build up base esations on the competition site.\\

DV 1.3 Data logger\\

DV 1.3.1 The officials will provide a standarized data logger that bust be installed in any DV during the competition. Further specifications for the data logger and required hardware and software interfaces can be found in the competition handbook.\\

DV 1.3.2 The intention of the data logger is to understand and reproduce the system state in case of failure. This includes a basic set of signals defined in the competition handbook and the set of vehicle-individual signals that have to be monitored by the Emergency Brake System (EBS) to ensure redundancy and fault detection.\\

DV 1.4 Remote Emergency System (RES)\\

DV 1.4.1 Every vehicle must be equipped with a standart RES specified in the competition handbook. The system consist of two parts, the remote control and the vehicle module.\\

DV 1.4.2 The RES must be purchased by the team.\\

DV 1.4.3 The RES has two functions:\\
	- When the remote emergency stop button is pressed, it must trigger the DV Shutdown Circuit (SDC) defined in DV 1.5.\\
	- Race-control-to-vehicle communication:\\
		- The race control can send a "Go" signal to the vehicle\\
		- The "Go" signal replaces green flags\\
		
DV 1.4.4 The RES vehicle module must be directly integrated in the vehicle's SDC with one of its relays hard-wired in series to the shutdown buttons.\\

DV 1.4.5 The antenna of the RES must be mounted unobstructed and without interfering parts in proximity (other antennas, etc.).\\

DV 1.5 Shutdown circuit\\

DV 1.5.1 The drivetrain-specific requirements for the SDC (see CV 4.1 or EV 6) remain valid for DV.\\

DV 1.5.2 If the SDC is opened by the Autonomous System (AS) or the RES, it hasto latched open by a non-programable logic that can only be reser manually (either a button outside of the vehicle, in proximity to the ASMS, or via LVMS power cycle).\\

DV 1.5.3 The SDC may only be closed by the AS, if the following conditions are fulfilled:\\
	- Manual Driving: Manual Mission is selected, the AS has checked that EBS is unavailable (No EBS actuation possible).\\
	- Autonomous Driving: Autonomous Mission is selected, ASMS is switched on and sufficient brake pressure is build up (brakes are closed).\\
	
DV 2 Autonomous System (AS)\\

DV 2.1 Signals\\

DV 2.1.1 Any signal of the AS is a SCS.\\

DV 2.2 Autonomous System Master Switch (ASMS)\\

DV 2.2.1 Each DV must be equipped with an ASMS, according to T 11.2\\

DV 2.2.2 The ASMS must be mounted in the middle of a completly blue circular area of >=50 mm diameter placed on a high contrasr background.\\

DV 2.2.3 The ASMS must be marked with "AS".\\

DV 2.2.4 The power supply of the steering and braking actuators must be switched by LVMS and ASMS.\\

DV 2.2.5 When the ASMS is in "Off" position, the following must be fulfilled:\\
	- No steering, braking and propulsion actuation can be performed by request os the autonomous system.\\
	- The sensors and the processing usits can stay operational.\\
	- The vehicle must be able to be pushed as specified in A 6.7.\\
	- It must be possible to operate de vehicle manually as normal CV or EV.\\

DV 2.2.6 It is strictly forbidden to switch the ASMS to the "On" position if a person is inside the vehicle.\\

DV 2.2.7 After switching the ASMS to the "On" position, the vehicle may not start moving and the brakes must remain closed ("AS ready" state, Figure 21) until a "Go" signal is sent via the RES ("AS driving" state, Figure 21).\\

DV 2.2.8 The ASMS must be fitted with a "lockout/tagout" capability to prevent accidental activation of the AS. The ASR must ensure that the ASMS is locked in the off position whenever the vehicle is outside the dynamic area or driven in manual mode.\\

DV 2.3 Steering Actuation\\

DV 2.3.1 Steeting system actuation (movement) must only happen if the vehicle is R2D.\\

DV 2.3.2 The steering system may remain active during an emergency brake maneuver while vehicle is in movement.\\

DV 2.3.3 Manual steering must be possible withoyt manual release steps (e.g. operating manual valves/ (dis-)connecting mechanical elements) while ASMS is switched "Off".\\

DV 2.4 Autonomous State Definitions\\

DV 2.4.1 The AS must implement the states end state transitions as shown in Figure 21.\\

DV 2.4.2 The AS must not have any other states or transitions.\\

DV 2.4.3 Numbered steps within an AS state machine transition (see Figure 21) must be checked in the given order. The vehicle must only perform a state-transition if all conditions are fulfilled. Until the transition is completed, the ASSIs must indicate the initialstate.\\

DV 2.4.4 The steering actuator can only have the following states:\\
	- "unavailable": power supply of the actuator is disconnected, manual steering is possible.\\
	- "available": power supply is connected and the actuator can respond to commands of the AS according to DV 2.3.1\\

DV 2.4.5 The service brake can only have the following states:\\
	- "unavailable": power supply of the actuator is disconnected, manual braking is possible.\\
	- "engaged": prevents the vehicle from rolling on a slope up to 15%.\\
	- "available": reswponds immediately to commands from the AS.\\
	For the state transition of the service brake actuator no manual steps (e.g. operating manual valves7(dis-)connecting mechanical elements) are allowed.\\
	
DV 2.4.6 The EBS can only have the following states:\\
	- "unavailable": the actuator is disconnected from the system/the energy storage is de-energized, emergency brake maneuver is not possible.\\
	- "armed": will initiate an emergency brake maneuver immediately if the SDC is opened or the LVMS supply is interrupted.\\
	- "activated": brakes are closed and power to EBS is cut. Brakes may only be released after performing manual steps.\\

DV 2.5 Autonomous System Status Indicators (ASSIs)\\

DV 2.5.1 The vehicle must include three ASSIs that must indicate the status of the AS (as defined in DV 2.4) correlating to illumination as shown:\\
	- AS Off: off\\
	- AS Ready: yellow continuous\\
	- AS Driving: yellow flashing\\
	- AS Emergency: blue flashing\\
	- AS Finished: blue continuous\\
	The ASSIs may not perform any other functions.\\
	
DV 2.5.2 One ASSI must be located on each side of the vehicle behind the driver's compartment, in a region 160 mm below the top of the main hoop and 600 mm above the ground. The third ASSI must be located at the rear of the vehicle, on the vehicle centreline, near vertical, 160 mm below the top of the main hoop and more than 100 mm above the brake light.\\

DV 2.5.3 Each ASSI must have a dark background and a rectangular, triangular or near round shape with a minimum illuminated surface of 15 cm^2. The ASSIs must be clearly vissible in very bright sunlight. When LED lights are used without a difuser, they may not be more than 20 mm apart. If a single line of LEDs is used, the minimum length is 150 mm. At least one ASSI must be visible from any angle of the vehicle.\\

DV 2.5.4 The state "AS Emergency" has to be indicated by an intermittent sound with the following parameters:\\
	- on-/off-frequency: 1 Hz to 5 Hz\\
	- duty cycle 50 %\\
	- sound level between 80 dBA and 90 dBA, fast weighting.\\
	- duration between 8 s and 10 s after entering "AS Emergency"\\
	The sound level will be measured with a free-field microphone placed free from obstructions in a radius of 2 m around the vehicle.\\
	
DV 2.6 Autonomous Missions\\
	
DV 2.6.1 The AS must at least implement the following missions:\\
	- Acceleration\\
	- Skidpad\\
	- Autocross\\
	- Trackdrive\\
	- EBS test\\
	- Inspection\\
	- Manual driving\\
	
DV 2.6.2 The inspection mission will be used during technical inspection while the vehicle is jacked up and all wheels are removed.\\

DV 2.6.3 The inspection mission is defined by slouly spinning the drivetrain and actuating the steering system with a sine wave. After25 s to 30 s the mission is finished and the transition to "AS Finish" must be initialized.\\

DV 2.6.4 The selected mission must be indicated by the Autonomous Mission Indicator (AMI).\\

DV 2.6.5 The AMI must be easy readable and can either be part of the dashboard or located next to the ASMS. If an e-ink display is used, it must be visible that the shown mission is up-to-date. AMI is considered SCS!\\

DV 2.7 Autonomous System Form (ASF)\\

DV 2.7.1 Prior ti the competition, all teams must submit a clearly structured documentation of their entire AS (including EBS and steering system) called ASF.\\

DV 2.7.2 The ASF must at least contain the following items:\\
	- All applied sensors (see also DV 4.2)\\
	- A clearly structured socumentation of the entire EBS.\\
	- A dbc file defining the supervised signals of the EBS monitoring.\\
	- A clearly structured documentation of the entire steering system.\\
	
DV 3 Emergency Brake System (EBS)\\

DV 3.1 Technical Requirements\\

DV 3.1.1 All specifications of the brake system from T6 remain valid.\\

DV 3.1.2 The vehicle must be equipped with an EBS, that must be supplied by LVMS, ASMS, RES and relay which is supplied by the SDC (parallel to the fuel pump relay).\\

DV 3.1.3 The EBS must only use passive systems with mechanical energy storage. Electical power loss at EBS must lead to a direct emergency brake maneuver (keep in mind T 11.3.1!).\\

DV 3.1.4 The EBS may be part of the hydraulic brake system. For all components of pneumatic and hydraulic EBS actuation not covered by T6, T9 is applied.\\

DV 3.1.5 When the EBS is part of the hydraulic brake system, the manual brake actuation (by brake pedal) may be deactivated for autonomous driving.\\

DV 3.1.6 The EBS must be designed so thar any official can easily deactivate it. All deactivation points must be in proximity to each other, easily accesible without the need for tools/removing any body parts/excessively bending into the cockpit. They must be able to be operated also when wearing gloves.\\

DV 3.1.7 A pictographic description of the location of the EBS release points must be clearly visible in proximity to the ASMS. The necessary steps to release the EBS must be clearly marked (e.g. pictographic or with a pull/push/turn arrow) at each releade point. This point must be marked by a red arrow of 100 mm length (shaft width of 20 mm) with "EBS release" in white letters on it.\\

DV 3.1.8 The use of push-in fittings is prohibited in function critical pneumatic circuits of the EBS and any other system which uses the same energy storage without proper decoupling.\\

DV 3.2 Functional Safety\\

DV 3.2.1 Due to the safety critical character of the EBS, the system must either remain fully functional, or the vehicle must automatically transition to the safe state in case of a single failure mode.\\

DV 3.2.2 The safe state is the vehicle at a standstill, brakes engaged to prevent the vehicle from rolling, and an open SDC.\\

DV 3.2.3 To get to the safe state, the vehicle must perform an autonomous brake maneuver described in section DV 3.3 and IN 6.3.\\

DV 3.2.4 An initial check has to be performed to ensure that EBS and its redundancy is able to built up brake pressure as expected. before AS transitions to "AS Ready".\\

DV 3.2.5 The tractive system is not considered to be a brake system.\\

DV 3.2.6 The service brake system may be used as redundancy if two-way monitoring is ensured.\\

DV 3.2.7 A red indicator light in the cockpit that is easily visible even in bright sunlight and clearly marked with the tettering "EBS" must light up if the EBS detects a failure.\\

DV 3.3 EBS Performance\\

DV 3.3.1 The system reaction time (the time between entering the triggered state and the start of the deceleration) must not exceed 200 ms.\\

DV 3.3.2 The average deceleration must be greater than 8 m/s^2 under dry track conditions.\\

DV 3.3.3 Whilst decelerating, the vehicle must remain in a stable drivind condition (i.e. no unintended yaw movement). This can be either a controlled deceleration (steering and braking control is active) or a stable braking in a straight line with all four wheeld locked.\\

DV 3.3.4 The performance of the system will be tested at technical inspection, see IN 6.3\\

DV 4 Sensors & Components\\

DV 4.1 Mounting\\

DV 4.1.1 All sensors and components must be securely mounted. For all mounts, T 8.3.1 applies.\\

DV 4.1.2 Sensors and components may not come into contact with the driver's helmet under any circumstances.\\

DV 4.1.3 All sensors and components must be positioned within the surface envelope (see T 1.1.16).\\

DV 4.1.4 Antennas that are exclusively acting as such with the longest side <100 mm may protrude from the envelope. For components behind the driver's compartment on a overhang by 25 % of their bounding box volume is accepted.\\

DV 4.1.5 Additionally, sensors may be mounted with a maximum distance of 500 mm above the ground and less than 700 mm forward of the front tyres (see figure 22). They must not exceed the width of the front axle (measured at the height of the hubs).\\

DV 4.2 Legal & Work Safety\\

DV 4.2.1 All sensors must fulfill the local legislative specifications (i.e. eye-protection classification for laser sensors, power limitation for radar sensors, etc.) in the conuntry of competition.\\

DV 4.2.2 This must be demonstrated by submitting the datasheet for the implemented sensors prior to the competition as an ASF Add Item Request (AAIR).\\

IN Technical Inspection\\

IN 1 General\\

IN 1.1 Technical Inspection Process\\

IN 1.1.1 The technical inspection is divided into the following parts:\\
	- Pre-Inspection\\
	- Mechanical Inspection\\
	- Driverless Inspection\\
	- Tilt Test\\
	- Vehicle Weighing\\
	- Noise Test\\
	- Brake Test\\
	- EBS Test\\

IN 6 Driverless Inspection\\

IN 6.1 Driverless Inspection Objective\\

IN 6.1.1 The objective of the DV inspection is to prove that:\\
	- All implementated sensors, including their mounting and location, are compliant with the rules.\\
	- RES, ASMS, EBS, ASSI end the datalogging system are working as specified.\\
	
IN 6.2 Driverless Inspection REquired Items\\

IN 6.2.1 The following items are required:\\
	- One ASRQ\\
	- The vechicle (in fully assembled, ready-to-race condition including mounted datalogger (see DV 1.3)\\
	- Data sheets for all perception sensors\\
	- Documents which proof that all perception sensors meet local legislation\\
	- RES remote control\\
	- ASF\\
	- Tools needed for the (dis)assembly of parts for DV inspection\\
	- Print.outs of rule questions (if aplicable)\\

IN 6.3 Driverless Inspection EBS Test\\

IN 6.3.1 The EBS performance will be tested dinamically and must demonstrate the performance described in DV 3.3.\\

IN 6.3.2 The test will be performed in a straight line marked with cones similar to acceleration.\\

IN 6.3.3 During the brake test, the vehicle must acelerate in autonomous mode up to at least 40 km/h within 20 m. From the point where the RES is triggered, the vehicle must come to a safe stop within a maximum distance of 10 m.\\

IN 6.3.4 In case of wet track conditions, the stopping distance will be scaled by the officials dependent on the friction level of the track.\\

IN 11 Brake Test\\

IN 11.1 Brake Test Procedure\\

IN 11.1.5 The EBS test (see IN 6.3) is conducted after all other elements of IN 11 have been passed.\\

IN 12 Post Event Inspection\\

IN 12.1 Post Event Inspection Procedure\\

IN 12.1.10 Directly after trackdrive os endurance and leaving the parc fermé, the data logger, se DV 1.3, will be disassembled from the vehicle.\\

S Static Events\\

S 2 Cost and Manufacturing Event\\

S 2.4 Bill of Materials (BOM)\\

S 2.4.4 The "systems" are:\\
	- Brake System\\
	- Engine and Drivetrain\\
	- Chassis and Body\\
	- Electrical\\
	- Miscellaneous, Fit and Finish\\
	- Steering System\\
	- Suspension System\\
	- Wheels, Wheel Bearings and Tires\\
	- Autonomous system\\
	
S 3 Engineering Design Event\\

S 3.1 Engineering Design Objective\\

S 3.1.3 For DV teams an evaluation concerning the capability of the vehicle to drive autonomously will also be part of this event. Therefore, all systems are required to drive autonomously will be investigated. This also includes a discussion about the hardware and the software used in the AS.\\

S 3.4 Autonomous Design Report (ADR)\\

S 3.4.1 The ADR will be uised to sort the teams into appropiate design queues, based on the quality of thos review.\\

S 3.4.2 The ADR should contain a description of the autonomous system with a review and derivation of the team's design objectives. Any information to scope, explain os highlight design features, concepts, methods or objectives to express the value and performance of the autonomous system to the judges shall be included at the team's discretion.\\

S 3.4.3 Evidence of information mentioned in the ADR should be brought to the competition and be available, on request, for review by the judges.\\

S 3.4.4 The ADR must not exceed five pages of content (text, which may include pictures and graphs).\\

S 3.4.5 Any portions of the ADR tha exceed five pages of content will not be evaluated.\\

S 3.4.6 The ADR must be written as a scientific paper.\\

S 3.5 Engineering Design Procedure\\

S 3.5.1 The design event starts with the submission of the DSS, the EDR, the ADR and their review respectively.\\

S 3.7 Engineering Desing Judging Criteria\\

S 3.7.1 The judger will evaluate the engineering effort based upon the team's DSS, EDR, and ADR, responses to questions and an inspection of the vehicle.\\

S 3.8 Engineering Design Scoring\\

S 3.8.1 The overal engineering design event maximum scoring is 150 points for CV/EV en 300 points for DV.\\

S 3.8.2 The maximum scores listed apply for the engineering design event:\\
	- Overall Vehicle Concept				75\\
	- Vechicle Performance					30\\
	- Mechanical / Structural Engineering	20\\
	- Tractive System / Powertrain			30\\
	- LV-Electrics / Electronics / Hardware	35\\
	- Autonomous Functionality				90\\
	- Engineering Design Report (EDR)		5\\
	- Autonomous Design Report (ADR)		15\\

D Dynamic Events\\

D 1 Dynamic Events General\\

D 1.1 Driver Limitations\\

D 1.1.4 DV teams need to register at least one driver for manual brake test but may register up to three drivers for testing in manual mode.\\

D 2 Driving Rules\\

D 2.1 Flags\\

D 2.1.2 There will be no flags signd for DV in autonomous mode.\\

D 2.2 Driving Under Powertrain\\

D 2.2.8 When driving autonomously, an ASR has to be present at the race control with the RES. Additionally, one single monitoring device (laptop, tablet, ...) may be brought (no complicated antenna construction or similar!).\\

D 2.4 Practice Track\\

D 2.4.2 A practice track for DV will be available (autonomous/manual).\\

D 2.5 Cones & Marking\\

D 2.5.1 Details of the cones used and more detailed track layout figures can be found in the competition handbook.\\

D 2.6 Startup-Procedure\\

D 2.6.1 No additional equipment (e.g. laptop, jack-up device, pressure tank, etc.) is allowed to start up the vehicle at the staging/starting line.\\

D 2.6.2 If the vehicle does not enter "AS Ready" stat within 1 min after being staged, the team may be sent to the preparation area by the officials.\\

D 2.6.3 The vehicle may only be staged with the steering system in straight position.\\

D 2.6.4 The vehicle may be pushed from the preparation area to the start line with activated LVS.\\

D 2.6.5 The EBS  may be armed already in the preparetion area.\\

D 2.6.6 The ASMS may only be switched on by the ASR after approval from an official at the starting line.\\

D 2.7 Vehicle Break Downs and Usage of RES\\

D 2.7.1 Stalling the engine or deactivating the tractive system for any reason during a dynamic event will result in Did Not Finished (DNF) as the autonomous system is nor allowed to restart the engine/reactivate the tractive system.\\

D 2.7.2 If a vehicle comes to standstill for any reason, it may have up to 30 s to attemptto continueto drive. If the vehicle doesn't restart within 30 s, it will be deactivated using RES, deemed disabled and scored as DNF for the run.\\

D 2.7.3 When the vehicle is driving in autonomous mode, one ASR must be present at the race control to operate the RES remote control.\\

D 2.7.4 The ASR or the officials may stop the vehicle using RES in any of the following cases:\\
	- Its behavior seems to be uncontrolled (e.g. driving off.course without visible intention to re-enter the track immediately).\\
	- It is mechanically or electrically damaged.\\
	- The average speed of the firt three laps in trackdrive (after completing the third lap) is beliow 2.5 m/s or the average speed of any of the following laps is below 3.5 m/s.\\
	- To ensure safe conditions on the track (e.g. persons or animals on the track). In this case the team will get a re-run.\\

D 2.7.5 If a vehicle breaks down or is stopped by the use of the RES it will be removed from the track, will not be allowe to re-enter the track and scored DNF.\\

D 2.7.6 If a traceable signal loss of the RES appears and doubtless proof can be brought by the team that it is was not self-inflicted, a re-run may be granted.\\

D 2.7.7 At direction of the officials, team members may be instructed to retrieve broken-down vehicles. This recovery may only be done under the control of the officials.\\

D 2.8 Procedure After Conpleting a Dynamic Event\\

D 2.8.1 The vehicle must be controlled by the ASR and an additional team member immediately after approval from the officials.\\

D 4.3 Skidpad Procedure\\

D 4.3.1 Each team has at least two runs. The dinal number of runs will be published besore the start of the event.\\

D 4.3.2 Starting order is based upon time of arrival. Teams on their first run will receive priority.\\

D 4.3.3 Staging - The foremost part of the vehicle is staged 15 m in fronto of the timekeeping line.\\

D 4.3.4 Starting - A go-signal from RES is used to indicate the pproval to begin.\\

D 4.3.5 The vehicle will enter perpendicular to the figure eight and will take one full lap on the right circle to establish the run. The next lap will be on the right circle and will be timed. Immediately following the second lap, the vehicle will enter the left circle for the third lap. the fourth lap will be on the left circle and will be timed. Immediately upon finishing the fourth lap, the vehicle will exit the track.\\

D 4.3.6 The vehicle will exit at the intersection moving in the same direction as entered and must come to a full stop within 25 m behiond the timekeeping line, inside the marked exit lane and enter the finish-state described in DV 2.5.\\

D 4.4 Skidpad Scoring\\

D 4.4.4 If a team's run time including penalties is below Tmax, additional points based on the following formula are given: [Scoring formila LaTex]\\
	Tteam is the team's best run time including penalties.\\
	Tmax is 1.5 times the time of the dastest vehicle including penalties.\\
	
D 5 Acceleration Event\\

D 5.1 Acceletarion Track Layout\\

D 5.1.2 The minimum track width is 3 m.\\

D 5.3 Acceleration Procedure\\

D 5.3.1 Each team has at least two runs. The final number of runs will be published before the start of the event.\\

D 5.3.2 Staging -  The foremost part of the vehicle is staged at 0.30 m behind the starting line. Vehicles will accelerate from a standing start.\\

D 5.3.3 Starting - A go-signal from RES is used to indicate the aproval to begin, timing starts only arter the vehicle crosses the starting line and stops after it crosses the finish line.\\

D 5.3.4 After the finish line, the vehicle must come to a full stop within 100 m inside the marked exit lane and enter the finish-state described in DV 2.5.\\

D 5.3.5 Starting order is based upon time arrival. Teams on their first run will receive priority.\\

D 5.4 Acceleration Scoring\\

D 5.4.3 If a team's best time including penalties is below Tmax, additional points based on the following formula are given: [Acceletation Scoring Formula LaTex]\\
	Tteam is the team's best time including penalties.\\
	Tmax is 1.5 times the time of the fastest vehicle including penalties.\\

D 6 autocross Event\\

D 6.1 Autoclross Track Layout\\

D 6.1.3 The autocross is using the same track as the trackdrive event (see D 8.1).\\

D 6.3 Autocross Procedure\\

D 6.3.1 There will be a trac kwalk prior the autocross. During the track walk no equipment (e.g. antennas, sensors, cameras, etc.) other than analog measurement devices (i.e. measurement wheel or measurement tape) is allowed.\\

D 6.3.2 Using data collected in a previous run is not permitted for the autocross event.\\

D 6.3.3 Each team has at least two runs consisting of one single lap. The final number of runs will be published before the start of the event.\\

D 6.3.4 The starting order is based on the time the team arrives at the autocross event. Teams on their first run will receive priority.\\

D 6.3.5 Staging - The vehicle is staged such that the front wheels are 6 m in front of the starting line on the track.\\

D 6.3.6 Starting - A go-signal from the RES is used to indicate the approval to begin. Timing starts after the vehicle crosses the starting line.\\

D 6.3.7 After the run the vehicle must come to a full stop within 30 m behind the finish line on the track and enter the finish-state described in DV 2.4.\\

D 6.5 Autocross Scoring\\

D 6.5.1 10 points are awarded to every team that finishes at least one run without DNF or DQ.\\

D 6.5.2 If a team's corrected elapsed time is below Tmax, points based on the following formula are given: [Autocross Score Formula]\\
	Tteam,i is the team's time including penalties of run i.\\
	Tmax is the time for driving the lap with 4 m/s.\\
	Tmin is the fastest corrected time of all teams.\\
	
D 8 Trackdrive and Efficiency Event\\

D 8.1 Trackdrive Tracklayout\\

D 8.1.1 The trackdrive layout is a closed loop circuit built to the following guidelines:\\
	- Straights: No longer than 80 m/s\\
	- Constant Turns: up to 50 m diameter\\
	- Harpin Turns: Minimum of 9 m outside diameter (of the turn)\\
	- Miscellaneous: Chicanes, multiple turns, decreasing radius turns, etc.\\
	- The minimum track width is 3 m/s\\

D 8.1.2 The length of one lap is approximately 200 m to 500 m.\\

D 8.2 Trackdrive Procedure\\

D 8.2.1 Starting order may be defined by the officials, based on previous dynamis event results.\\

D 8.2.2 Before starting a run, each DV, with a fuel tank, must be filled to the fuel level line (see CV 2.6.3, "Fuel Level Line") at the fueling station. During the fueling, once filled to the described line, no shaking or tilting of the tank ,the fuel system or the entire vehicle is allowed.\\

D 8.2.3 There will be a maximum of two runs, each run consisting of ten laps, The number of runs and the starting order will be announced before the start of the event.\\

D 8.2.4 Staging - The vehicle is staged such that the front wheels are 6 m in front of the starting line on the track.\\

D 8.2.5 Starting - A go-signal from RES is used to indicate the approval to begin. Timing starts after the vehicle crosses the starting line.\\

D 8.2.6 After ten laps the vehicle must come to a full stop within 30 m behind the finish line on the track and enter the finish-state described in DV 2.4.\\

D 8.2.7 There will be no last lap signal i.e. the vehicle should count laps itself.\\

D 8.2.8 The team must proceed directly to the fueling station.\\

D 8.3 Trackdrive Scoring\\

D 8.3.1 If there is more than one run per vehicle, the run with the highest score of trackdrive is scored.\\

D 8.3.2 Tmax and Tmin for the trackdrive and efficiency score are calculated based on all valid runs.\\

D 8.3.3 Each lap of the trackdrive event is individually timed. The corrected elapsed time is determined by adding any penalty times.\\

D 8.3.4 If a team's corrected elapsed time is below Tmax and the run was not  DNF or DQ, points based on the following formula are given: [Trackdrive score formula]\\
	Tteam is the team's corrected elapsed time.
	Tmax is 2 times of the corrected  elapsed time of the fastest vehicle over all runs.\\

D 8.3.5 An additional five points are awarded for every completed lap, independent of the corrected elapsed time. This is also applied for teams that do not finish the trackdrive i.e. get a DNF.\\

D 8.4 Efficiency Scoring\\

D 8.4.1 Energy efficiency is measured during the trackdrive event.\\

D 8.4.2 Only vehicles which complete the trackdrive event receive points for efficienty.\\

D 8.4.3 Efficiency is scored for the run with the highest trackdrive score.\\

D 8.4.4 Teams whose uncorrected elapsed endurance time exceeds 2 times of the uncorrected elapsed time of the fastest vehicle over all runs receive 0 points for efficiency.\\

D 8.4.6 Rules D 7.9.3, D 7.9.5 and D 7.9.8 are applied.\\

D 8.4.7 The trackdrive energy is calculated based on the following formula: [Energy formula]\\
	Vteam is the team's corrected used fuel volume.\\

D 8.4.8 The ream's efficienty factor is calculated based on D 7.10.6.\\

D 8.4.9 Efficiency points are calculated using the following formula: [efficiency scoring formula]\\
	Eteam is the team's efficienty factor.\\
	Emax is the highest efficiency factor of all teams who are able to score points in efficiency.\\

D 9 Dynamic Events Penalties\\

D 9.1 General Penalties\\

D 9.1.3 Cones that are DOO are not replaced/reset during the run. There will be no re-run due to cones in the driving path or disorientation due to missing cones.\\

D 9.1.6 An Unsafe Stop (USS) is defined as not stopping within the specified area and/or not entering the finish-state described in DV 2.4.\\

Puntos relacionados:\\
A 2.3.1 - A 2.2\\
A 2.3.2 - T 3.4.3, T 3.7.4, T 3.9.5, T 4.5.1, T 4.5.3\\
A 4.9.2 - A 4.8\\
A 4.9.4 - A 4.3\\
A 5.1.1 - Handbook (documents and forms to be submitted)\\
A 5.6.3 - A 6.4, T 13.3, DV 3, Handbook (video size limitation)\\
A 6.7.4 - DV 2.2, DV 2.2.8\\
T 4.2.4 - T 4.2.1\\
T 4.2.5 - T 4.3.4, T 4.2.4\\
T 6.1.4 - T 6.1.1\\
T 11.9.1 - CV 4.1, EV 6.1, DV 3.2.7\\
T 13.5.1 - DV 4\\
CV 1.2.2 - T 11.3, DV 2.2\\
DV 1.1 - T, CV, EV, A 2.3\\
DV 1.2.2 - DV 1.4\\
DV 1.3.1 - Handbook (data logger)\\
DV 1.4.1 - Handbook (RES)\\
DV 1.4.3 - DV 1.5\\
DV 1.5.1 - CV 4.1, EV 6\\
DV 2.2.1 - T 11.2\\
DV 2.2.5 - A 6.7\\
DV 2.4.4 - DV 2.3.1\\
DV 2.5.1 - DV 2.4\\
DV 2.7.2 - DV 2.4\\
DV 3.1.1 - T6\\
DV 3.1.3 - T 11.3.1\\
DV 3.1.4 - T6, T9\\
DV 3.2.3 - DV 3.3, DV 6.3\\
DV 3.3.4 - IN 6.3\\
DV 4.1.1 - T 8.3.1\\
DV 4.1.3 - T 1.1.16\\
IN 6.2.1 - DV 1.3\\
IN 6.3.1 - DV 3.3\\
IN 11.1.5 - IN 6.3\\
IN 12.1.10 - DV 1.3\\
D 2.5.1 - Handbook (cones)\\
D 4.3.6 - DV 2.5\\
D 5.3.4 - DV 2.5\\
D 6.1.3 - D 8.1\\
D 6.3.7 - DV 2.4\\
D 8.2.6 - DV 2.4\\
D 8.4.8 - D 7.10.6\\
D 9.1.6 - DV 2.4\\

\end{document}
